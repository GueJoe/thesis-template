%!TEX root = thesis.tex

\chapter{Abkürzungsverzeichnis}
%nur verwendete Akronyme werden letztlich im Abkürzungsverzeichnis des Dokuments angezeigt
%Verwendung: 
%		\ac{Abk.}   --> fügt die Abkürzung ein, beim ersten Aufruf wird zusätzlich automatisch die ausgeschriebene Version davor eingefügt bzw. in einer Fußnote (hierfür muss in header.tex \usepackage[printonlyused,footnote]{acronym} stehen) dargestellt
%		\acs{Abk.}   -->  fügt die Abkürzung ein
%		\acf{Abk.}   --> fügt die Abkürzung UND die Erklärung ein
%		\acl{Abk.}   --> fügt nur die Erklärung ein
%		\acp{Abk.}  --> gibt Plural aus (angefügtes 's'); das zusätzliche 'p' funktioniert auch bei obigen Befehlen
%	siehe auch: http://golatex.de/wiki/%5Cacronym
%	
\begin{acronym}
	
	%A
	\acro{API}{Application-Programming-Interface}
	%B
	%C
	%D
	%E
	%F
	%G
	%H
	%I
	\acro{ISO}{Internationale Organisation für Normung}
	\acroplural{ISO}[ISO]{Internationalen Organisation für Normung}
	%J
	%K
	%L
	%M
	%N
	%O
	%P
	%Q
	%R
	%S
	%T
	%U
	%V
	%W
	%X
	%Y
	%Z
		
\end{acronym}


